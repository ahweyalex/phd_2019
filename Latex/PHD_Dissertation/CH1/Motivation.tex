  
% high lights of overview purpose 
As large-scale manufacturing increases, rapid identification techniques have been developed. Among these identification techniques, automatic identification (Auto-ID) procedures  has become popular in many fields of industries. Some of the most commonly used auto-ID procedures include the following but not limited to bar-code system, optical character recognition (OCR), bio-metric MM, and Radio frequency identification (RFID). This dissertation will focus on RFID technology.   

% define RFID and its gen purpose
Radio-frequency identification (RFID) technology uses electromagnetic fields to enable automatic identification of uniquely tagged objects or specimens at a distance. This technology is used to access control systems, automatic toll collection systems, vehicle tracking and immobilizers, ID and security, inventory tracking, and bio-logging. This research will focus on inventory tracking and bio-logging.

% go into details on inventory 
RFID technology is being quickly adapted into many companies' supply chains. This is due to RFID being cost-effective and providing advantages to many of the other Auto-ID procedures. Specifically, RFID technology has the advantage over bar-code and optical type systems that it does not have to be within optical line of sight.  
%RFID does not require the inventory to be in line of sight, can be obscured, and track multiple at once. % RUYLE: problematic statement
RFID systems operates in two ranges, near-field and far-field back-scatter. For the application of tracking large quantities companies will use the far-field back-scatter. The far-field back-scatter will be discussed in more detail in Section 1.2.3.  

As RFID technology continues to grow within supply chains, the tracking and identifying of various inventory needs to be taken into account. However, most RFID antennas (tags) are ``dipole-like" loops, folded dipoles, or bulky microstrip. These antennas perform poorly when applied to a metallic object. Previous work has shown an RFID antenna that is placement insensitive, no impedance change with attached to different materials \cite{ruyle2012small} \cite{ruyle2016placement}. The realized gain of this RFID tag is quite low-limiting read range \cite{ruyle2016placement}. However, it was also fund in previous work that the RFID tag couples into metallic object that is attached to\cite{moreno2016RzGain}.This dissertation will investigate how metallic object that the RFID tag is being used to track can be used strategically to increase read-range. 

% go into details on bio-logging 
Bio-logging is the data collecting of animal behavior and movement. Traditionally, the bio-logging of specimens can be a long, tedious, inaccurate, and intrusive task for researchers. Recent RFID advancements have enabled researchers to utilize RFID technology for bio-logging. RFID brings forth many benefits, such as minimal disturbance to subjects, fully automatized measurements, full-time monitoring, and inexpensive tags. However, current commercial available RFID technology for readers is expensive, not setup for animal tracking,  proprietary equipment, and difficult to apply modifications to hardware and software. Previous work by Dr. Bridge from University of Oklahoma has developed an open-source inexpensive modifiable RFID tag system. This system is called the Electronic Transponder Analysis Gateway (ETAG) Reader. This RFID system is a near-field system withe the RFID tag being passive transponder (tag). The system is a modified Arduino M0, programmable circuit board, a popular for the ease of use and available documentation. The ETAG audience do not have background in electronics or antenna design. 

%%%% REWRITE %%%
% rewrite to have stronger scientific impact 
%Therefore, this dissertation will review the background required in designing the antenna reader for a near-field RFID system, provide a software antenna design tool, and improved antenna readers via the software tool.   

%