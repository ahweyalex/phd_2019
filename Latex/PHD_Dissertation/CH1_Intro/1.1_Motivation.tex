
This dissertation investigates the backscattering within two implementations, radio frequency identification (RFID) and radar cross section (RCS). The RFID is associated with biological research, specifically the antenna design for bio-logging. 

Bio-logging is the data collecting of animal behavior and movement. The bio-logging of specimens can be a long, tedious, inaccurate, intrusive tasked that the researcher has to part take. Recent advancements in RFID  is enabling these researchers to utilize this technology. The benefits that RFID have upon this area of research are minimal disturbance to subjects, automatized and full-time monitoring, and low cost tag. However, current commercial available RFID technology tend to be expensive, not setup for animal tracking, difficult to apply modifications such as hardware and software, and proprietary equipment. There has been research and development of an open-source inexpensive modifiable RFID tag system. This system is called  Electronic Transponder Analysis Gateway (ETAG) Reader developed via Dr. Bridge from University of Oklahoma. The ETAG Reader is near-range via a passive transponder (tag), the system is a modified Arduino M0. Arduino have a lot of built functionality and documentation.      

 Previous work has shown a direct relationship between RFID tag's realized gain when placed upon a metallic object. This relationship indicates as the metallic object electrically dimensions is increases so does too the tag's realized gain. There is a quantitative relationship between antenna gain and antenna RCS due to its re-radiation properties. \cite{sinclair1947measurement} \cite{king1949measurement}  \cite{appel1979accurate}  Thus this will present the finding The other backscattering investigation involves the impact on RCS of metallic object with a placement of RFID tag.