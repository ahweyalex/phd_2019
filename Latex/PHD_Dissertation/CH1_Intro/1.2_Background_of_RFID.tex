As large-scale manufacturing keeps increasing to meet population demand, rapid identification techniques have been developed. Among these identification techniques, automatic identification procedures (Auto-ID) has become popular in many fields of industries, biological research, manufacturing, etc. Some of the most commonly used auto-ID procedures include the following but not limited to barcode system, optical character recognition (OCR), bio-metric MM, and Radio frequency identification (RFID). This dissertation will focus on RFID technology.   

RFID technology uses electromagnetic fields to enable its users' to automatically identify uniquely tagged specimens at a distance wirelessly without a line of sight. Current use of this technology includes tracking applications of crates and pallets, access control systems, automatic toll collection systems, vehicle tracking and immobilizers, ID and security, etc. The bulk of the research conducted in this dissertation is associated with biological research, this will be further discussed in Section 1.4 Research Objectives. 

\newpage
An RFID system is composed of three components:
\begin{itemize}
    \item A transponder (tag), generally a construction of antenna, chip, and sometimes a battery. It is the data-carrying device.
    \item An interrogator (the reader/write device), generally a construction of an antenna, an RF electronics, and control module
    \item A controller (host electronic or computer), generally hosts the database 
\end{itemize}

Some RFID systems can have the interrogator and the controller be one component. The simple interaction between these components is that the tag and interrogator communicate information between each other via radio-frequency (RF) waves. This interaction occurs when the tagged specimen enters the read zone of the interrogator, retrieves information from the tag. The information that tags can store can be serial numbers, time stamps,  or other useful data. Once the interrogator retrieves the tag's information the controller can implement another procedure such as recording inventory, granting access, active another electronic such as a motor, etc. 

The RFID tags fall into two categories, passive and active transponders. Passive RFID tags do not have on-board power supply instead they obtain their power from the interrogator's transmitted signal (magnetic or
electromagnetic field). The tag transmits its data via modulation (e.g. by load modulation or modulated backscatter). In most cases, passive tags are smaller and less expensive than active tags. Active tags contain an on-board power supply, such as a battery or solar cell, to provide voltage to the chip. The active tags can generate their own fields and modulation thus increasing the read-range between the transponder and the interrogator. Important to note that active tags are typically not able to generate high-frequency signals alone, can only modulate the fields from the interrogator.    

Tags can also be categorized as read-only (RO) and read/write (RW). The RO tags can only be read. Once these RO tags are fabricated their data can be altered. Typically used for static information such as part numbers, ID, serial numbers, etc. The RW provides more flexibility such as allowing the data to be changed, storing much more information than RO tags, and easy accessibility.   

As previously mentioned the tag transmits its data via modulation. The two modulations are load modulation or modulated backscatter. The load modulation uses the near-field coupling between reader and tag, can be describe via Faraday's principle. An alternating current is passed through the reader's (coiled) antenna, generating a magnetic field. Once the RFID tag is placed within the reader's fields this will result in an induced alternating voltage across the tag. This voltage will be used within the tag to turn its chip. Once the chip has been successfully activated, the tag yields its own magnetic fields that interact with the reader. The tag's load is applied to the reader's coiled antenna over time, causing a modulation. Typically load modulations operating under 100MHz (LF and HF).

This modulation can be encoded to retrieve the tag's data. The modulated backscatter (or far-field coupling) retrieves the electromagnetic waves from the tag's antenna when not within the near-fields. These tags are typically designed for a frequency band such that there is a mismatch causing energy to be reflected back to the reader. The tag's impedance can be changed over-time to cause modulation, this modulation can be encoded by the reader to obtain the tag's data. Typically modulated backscatter operates within the UHF frequency band. 

A brief mentioned two basic modulation techniques used within RFID is amplitude shift keying (ASK) and binary phase-shift keying (BPSK). ASK represents digital data as variations in the amplitude of a carrier wave. The binary symbol 1 is represented as a duration of T seconds at a fixed-amplitude and binary symbol is for all other cases that are not that fixed-amplitude. BPSK (or PRK or 2PSK), a form of phase-shift keying, is a two-phase modulation scheme. The signal has two different phase states in the carrier signal, $\theta=0^\circ$ (binary 1) and $\theta=180^\circ$ (binary 0). In Figure [] shows an example of these two modulation schemes.




