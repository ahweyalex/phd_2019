Similarly, as RFID's backscatter (or far-field coupling), radar cross section (RCS) is backscatter procedure. RCS is a measurement of the reflective strength of a target defined as 4$\pi$ times the ratio of the power unit solid angle scattered in a specified direction to the power unit per unit area in-plane wave incident on the scatterer from a specified direction as defined in \cite{jay1977standard}. A simplified explanation is an electromagnetic wave that is transmitted from a source, the wave interacts with the target thus the wave is absorbed and scattered, the return scattered waves to the receiver are measured..

However, an important distinguished between RFID and RCS implementation of backscatter is that RFID backscatter contains a modulated signal (i.e. information associated with binary data) and RCS is associated with the field strength measured upon the backscatter. RCS obtains information relating to the target's electrically geometrical and electrical material properties.

The system that is used to take RCS measurements are constructed of 4 subsystems: 
An RFID system is composed of three components:
\begin{itemize}
    \item  A transmitter, subsystem that creates the waveform to be transmitted via the transmit antenna
    \item Antenna(s), front end devices that transmit and/or receives the scattered energy   
    \item  A receiver, subsystem that detects and signal processes the received scatter energy
    \item An indicator, subsystem to display the measured data. Typically the data is converted into In-phase(I) and Quadrature-phase(Q) signals.
\end{itemize}

The I signals are created via mixing the received signal and a local oscillator (LO) signal that is phase coherent with the transmitted signal. The Q signals mixed with a $90^\circ$ phased shifted version of the received signal with the LO. Many current systems have an Analog-to-digital (A/D) converters to convert the measurements into digital  for signal processing.  


The signals transmitted in RCS can be a continuous wave (CW) or signal pulsed. CW radar setup is more simple than the pulsed setup. However, unmodulated CW provides no range information. The general setup for a CW radar is a CW signal that is transmitted when the target object's scatter signal is measured. This setup allows the user to calculate the object's doppler shift in post-processing. CW radar provides the highest average power for a given peak source power. Frequency-modulated continuous wave (FM-CW) radar systems vary its operating frequency during measurements. This allows for improved tracking target's range and relative velocity. 

Pulsed radar is the more commonly used setup for RCS. This RCS system transmits a continuous train of pulses at a certain pulse repetition frequency (PRF) where the signal's amplitude is modulated. Once the pulse reaches upon the target the signal is scattered, backscattered measured by the RCS system  is detected and displayed. This setup allows the user to obtain range information. This is possible due to pulsewidth ($\tau$) which is defined as the amount of time allotted for the transmitted pulse.     